%!TEX root = ../dissertation.tex

\chapter{Embedded reconstruction of local primitives}
\label{chp:retina}
\epigraph{We here propose to do just what Copernicus did in attempting to explain the celestial movements. When he found that he could make no further progress by assuming that all the heavenly bodies revolved round the spectator, he reversed the process and tried the experiment of assuming the spectator revolved, while the stars remained at rest.}{Immanuel Kant, \textit{Critique of Pure Reason}}
\textit{Iniziare la ricostruzione a tempo 0 in maniera embedded durante il readout}

\section[The Field Programmable Gate Array]{The Field Programmable Gate Array $\bigl($FPGA$\bigr)$}

\section{The RETINA Project}
In the context of advancing technologies for real-time track reconstruction, the INFN-RETINA project has emerged with a focus on implementing an FPGA-based computing architecture inspired by  mammalian brain early image reconstruction processes. Progressing beyond its initial research phase, for Run3 operations the project has now developed a demonstrator on real hardware for reconstructing data from the Vertex Locator (VELO), LHCb's pixel detector. The initial phase involving the reconstruction of particle hits out of the raw pixel data, is performed by means of a 2D cluster-finding algorithm implemented in the 52 readout FPGAs of the VELO, for a total of 104 parallel channels. Each channel further subdivides clusters among multiple parallel sub-channels. Utilizing a pipelined internal architecture, certain operations are executed swiftly "on the fly" for all hits detected in each event, without causing delays in data acquisition or requiring extensive computational resources, provided the firmware is appropriately designed. Notably, this efficiency is maintained despite an average hit rate of approximately $4 \cdot 10^{10}$/s, which may appear prohibitive for exhaustive processing.\\

\section{The Clustering}
\textit{description of the clustering algorithm and a brief description of its implementation on FPGA\\
}
\subsection{Algorithm}
\subsection{Firmware implementation}

\section{The VELO Counters}



