%!TEX root = ../dissertation.tex
For the ongoing LHC Run~3, LHCb has introduced an innovative trigger system that employs full software reconstruction of every collision event, marking the first time that an average \SI{30}{\mega\hertz} stream of LHC collision events is filtered based on offline-quality event reconstruction performed in real time. This system processes a data flow of approximately \SI{40}{\tera\bit\per\second}, posing significant computational challenges. To address this, LHCb adopted a heterogeneous computing system utilizing CPUs, GPUs, and FPGAs simultaneously. As LHCb plans to introduce additional computing power at an even earlier stage for Run~4, a 2D FPGA based cluster finding algorithm was developed and is already operational in Run~3. This algorithm determines the coordinates of all hits in the VELO -- the LHCb pixel detector for vertex reconstruction -- at the full collision rate of \SI{30}{\mega\hertz}.

The goal of this thesis is to explore what crucial measurements are enabled by making the flow  of $\sim4 \times 10^{10}$ hits per second on the VELO available in real time. I focused on applications that could be practically implemented with the limited residual processing power already available within the Run~3 LHCb readout system, ensuring no negative impact on throughput. To this aim, I used simple statistical methods, such as counting rates of reconstructed hits in specific VELO regions.

By implementing a number of appropriate counters and statistically combining them, I evaluated seven linear estimators: a luminosity estimate, the average position of the luminous region in the transverse plane ($x$ and $y$ coordinates), and the average positions of the two VELO halves in both transverse components. The luminosity measurement was calibrated using a van der Meer scan and a trimmed mean of the different luminosity counters, achieving a statistical precision of $0.3\%$, which is currently the best available in real time at LHCb.

Leveraging Principal Component Analysis, I developed a method to analyse the stream of data with the aim of monitoring the luminous region and VELO positions. This process involves a simple scalar product between the cluster counters and weights estimated from Monte Carlo simulations, allowing fast and accurate measurements with a resolution of \SI{4}{\micro\meter} every few milliseconds. 

The resolution of the VELO position estimators is measured to be between \SI{6}{\micro\meter} and \SI{11}{\micro\meter}. 

All the measurements discussed in this thesis are already implemented in the online monitoring tool of LHCb and currently provide immediate feedback to the experiment for the quantities discussed above.

These results demonstrate the significant benefits of high-throughput heterogeneous computing at the early stages of data processing, achieved in this context by using specialised computing devices. The results obtained in this thesis encourage further exploration of these technologies in future experiments.