%!TEX root = ../dissertation.tex
%\normalsize

%“È Maya, il velo ingannatore, che offusca lo sguardo dell’individuo incolto; a costui, invece della cosa in sé, si mostra solo il fenomeno nel tempo e nello spazio; avvolge gli occhi dei mortali e fa loro vedere un mondo del quale non può dirsi né che esista, né che non esista; perché ella rassomiglia al sonno, rassomiglia al riflesso del sole sulla sabbia, che il pellegrino da lontano scambia per acqua; in questa forma la sua conoscenza limitata non vede l’essenza delle cose, che è una sola, ma i suoi fenomeni, che sono distinti, separati, innumerevoli, molto diversi gli uni dagli altri, addirittura contrapposti.\\
%\footnotesize
%--- Arthur Schopenhauer, \textit{Il Mondo come Volontà e Rappresentazione}\\

\epigraph{If I must speak, my noble Memmius, as nature's majesty now known demands}{Lucretius, \textit{De Rerum Natura}}

%\begin{flushright}
%``If I must speak, my noble Memmius,\\
%As nature's majesty now known demands"\\
%\footnotesize
%--- Lucretius, \textit{De Rerum Natura}\\
%\end{flushright}
%\normalsize
